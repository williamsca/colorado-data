\documentclass[11pt,article,oneside]{memoir}
% \input{vc}

\usepackage{booktabs, amsmath, memoir-article-styles, graphicx, hyperref}

% packages for 'etable' output from R
\usepackage{longtable, makecell, rotating}

\usepackage[authoryear]{natbib} % citation manager

\usepackage{tikz}
\usetikzlibrary{positioning}

\usepackage[dvipsnames]{xcolor} % Standard suite of named colors

\hypersetup{
    colorlinks=true, % Colored links instead of boxed
    linkcolor={RoyalBlue}, % Color of internal links
    citecolor={RoyalBlue}, % Color of citations
    urlcolor={Cerulean} % Color of URL links
}

\usepackage{float}
\floatplacement{figure}{H} % control figure placement

\setlength{\parindent}{20pt} % Sets paragraph indentation
\setlength{\parskip}{5pt plus 0pt} % Sets paragraph spacing





 

%\author{true}

\author{\Large Colin
Williams\vspace{0.05in} \newline\normalsize\emph{University of
Virginia} \newline\footnotesize \protect\url{chv7bg@virginia.edu}\vspace*{0.2in}\newline }


\date{}


\begin{document}  
\setkeys{Gin}{width=1\textwidth} 	
%\setromanfont[Mapping=tex-text,Numbers=OldStyle]{Minion Pro} 
%\setsansfont[Mapping=tex-text]{Minion Pro} 
%\setmonofont[Mapping=tex-text,Scale=0.8]{Pragmata}
\chapterstyle{article-3} 
\pagestyle{kjh}

\published{.}




\hypertarget{introduction}{%
\section{Introduction}\label{introduction}}

\hypertarget{house-price-index-analysis}{%
\section{House Price Index Analysis}\label{house-price-index-analysis}}

\hypertarget{methods}{%
\subsection{Methods}\label{methods}}

We analyze the effect of effective assessment ratios on house price
indices using instrumental variables estimation. The house price index
(HPI) data comes from Bogin, Doerner, and Larson and provides
county-level price indices for Colorado counties from 1975-2024. We
merge this data with our county-year panel of mill levies and assessment
ratios for the overlapping period of 1980-1995.

Our identification strategy exploits exogenous variation in effective
assessment ratios induced by Colorado's Gallagher Amendment.
Specifically, we instrument for the effective assessment ratio using a
measure based on predetermined (1980) residential valuation shares
interacted with time-varying residential assessment rates. This
instrument captures the mechanical effect of Gallagher on assessment
ratios while being orthogonal to unobserved factors that might
independently affect house prices.

We estimate

\[y_{ct} = \alpha + \beta \text{EAR}_{ct} + \lambda_c + \delta_t + \epsilon_{ct}\]

where \(y_{ct}\) is the log house price index for county \(c\) at time
\(t\), \(\text{EAR}_{ct}\) is the effective assessment ratio,
\(\lambda_c\) are county fixed effects, and \(\delta_t\) are year fixed
effects. The parameter of interest is \(\beta\), which captures the
effect of effective assessment ratios on house prices.

\hypertarget{results}{%
\subsection{Results}\label{results}}

Table \ref{tab:hpi-effects} presents both OLS and instrumental variables
estimates of the effect of effective assessment ratios on log house
prices. The OLS estimate suggests a small negative but statistically
insignificant relationship (-0.311, standard error 0.356). The IV
estimate is larger in magnitude (-2.00, standard error 2.09) but remains
statistically insignificant at conventional levels.

\begin{table}[htbp]

\caption{The effect of property tax rates on housing prices}\label{tab:hpi-effects}
\input{results/tables/hpi-eff_ar-iv.tex}
\begin{footnotesize}
\begin{flushleft}
\emph{Note:  } The table 
\end{flushleft}
\end{footnotesize}


\end{table}


% original bibliography code
% %
\end{document}